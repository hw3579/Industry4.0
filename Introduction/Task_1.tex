\section{Task 1}
\FloatBarrier % Now figures cannot float above section title


\subsection{How original and unique is the designed robot?}

After discussion and doing some references, our group decide to make our robot arm as a four-arm robot and define the robot arms and joints, because there are some advantages of four-arm robot:

\begin{itemize}
\item better accuracy and stability of the movement;
\item bigger working arrangement;
\item better loading allowance;
\item better collaboration.
\end{itemize}


\textbf{And there are also some unique assessments of our four-robot arms can meet:}

\begin{itemize}
\item The welding robot arm allows more precise control of the welding position and angle.
\item The stability in high temperature, low temperature and hazardous gas environments ensure safe and stable welding.
\item Compared to manual welding, welding robots can automate welding work, thus increasing productivity and reducing labor costs.
\item To make sure the arms and joints will not clash into each other.
\item Risk of joint interference: Do kinematic analysis to determine the minimum spacing and range of the movements between the joints and the robot arms, so that we can avoid the interference of the clash of the joints.
\item Simulation analysis: After setting the exact parameters of the arms and joints, we use MATLAB to do simulation of real situations, and make sure the clash of the joints and arms won't happen.
\item Design concept: After doing some researches of designing the robots, we consider the working arrangement, the stability of the robots and its operability. In order to achieve the function of welding in an exact area, we finally make the definitions of the parameters of all the arms and joints.
\end{itemize}


\subsection{All the arms and the joints:}

\textbf{robot arms: [name, mass, center of mass, Inertia]}


\begin{minipage}[htbp]{\textwidth}
    \makeatletter\def\@captype{table}
    \centering
    \scalebox{1}{
    \begin{tabular}{cccc}
    \hline
    Name & Body Mass (kg) & Center of mass & Inertia ($I_{xx}$ $I_{yy}$ $I_{zz}$) ($kg \cdot m^2$)                  \\ \hline
    R1   & 10        & (0 0 0)        & (0.27 0.27 0.8 )     \\
    R2   & 10        & (0 0 0)        & (0.27 0.27 0.8 )     \\
    P1   & 1.5       & (0 0 0)        & (0.07 0.07 0.07 )    \\
    P2   & 1.5       & (0 0 0)        & (0.07 0.07 0.07 )    \\
    Tool & 1.2       & (0 0 0)        & (0.002 0.002 0.004 ) \\ \hline
    \end{tabular}} 


    \caption{Experiment parameters}
    \label{t1} 
\end{minipage}


\textbf{Joint angles:(The limit of the joint, Type of the joints)}

\begin{minipage}[htbp]{\textwidth}
    \makeatletter\def\@captype{table}
    \centering
    \scalebox{1}{
    \begin{tabular}{cccc}
    \hline
    Joint & Type & Position Limit (rad \& m) & Joint Axis                   \\ \hline
    1   & revolute       & $[-5\frac{\pi}{180}, 5\frac{\pi}{180}]$ & [0 0 1]     \\
    2   & revolute       & $[-30\frac{\pi}{180}, 30\frac{\pi}{180}]$        & [0 1 0]     \\
    3   & prismatic      & [-0.5,0.5]        & [1 0 0]    \\
    4   & prismatic      & [-1, 1]        & [0 1 0]    \\ 
    Fixed &revolute & N/A & N/A \\\hline
    \end{tabular}} 

    (Unit: mm)
    \caption{Experiment parameters}
    \label{t1} 
\end{minipage}




\iffalse
The frame is made of mild steel and has a uniform rectangular cross-section. Also, The rig is equipped with two gauges to monitor the horizontal deflection of the beam and its central vertical deflection. The yield strenth of the steel is 250MPa.

The experimental parameters are shown in the following table \autoref{t1}.

\begin{minipage}[htbp]{\textwidth}
    \makeatletter\def\@captype{table}
    \centering
    \scalebox{1}{
        \begin{tabular}{lllll}
            \hline
            Data type &Height & Length & Thickness & Width \\ \hline
            Theory&200    & 300    & 3.00      & 12.00 \\ 
            Actual&200    & 304    & 3.27      & 12.97 \\ \hline
            \end{tabular}} 

    (Unit: mm)
    \caption{Experiment parameters}
    \label{t1} 
\end{minipage}


The experiment entails the following steps:
\begin{enumerate}
    \item Measure and record the dimensions of the frame and its cross-section.

    \item Ensure the loading rig is in proper working condition and inspect cables for damage.
    
    \item Zero the force and displacement readings while the rig is still unloaded.
    
    \item Gradually apply increasing horizontal and vertical loads to the frame in increments of 10 N. The relationships between P and W are $y=x$\label{ee1}.
    
    \item Record the applied forces and corresponding deflections for each increment.
    
    \item Continue the process until a plastic failure occurs, and observe the formation of plastic hinges and position.
    
    \item Unload the frame and identify the locations of plastic hinges by observing permanent rotation in the joints.
    
\end{enumerate}

Here are the recorded data \autoref{t2}.

\begin{minipage}[htbp]{\textwidth}
    \makeatletter\def\@captype{table}
    \centering
    \scalebox{0.85}{
        \begin{tabular}{ccccccccccccccc}
            \hline
            \multirow{2}{*}{Force(N)}    & Vertical   & 10   & 20   & 30   & 40   & 50   & 60   & 70   & 80   & 90   & 100  & 110  & 120   & 130   \\
                                      & Horizontal & 10   & 20   & 30   & 40   & 50   & 60   & 70   & 80   & 90   & 100  & 110  & 120   & 130   \\ \hline
            \multirow{2}{*}{Distance(mm)} & Vertical   & 0.06 & 0.05 & 0.1  & 1.42 & 2.13 & 2.26 & 2.8  & 3.22 & 3.62 & 4.28 & 5.23 & 7.46  & 12.88 \\
                                      & Horizontal & 0.88 & 1.58 & 2.15 & 2.99 & 4.04 & 4.44 & 5.06 & 6.48 & 7.63 & 8.55 & 9.31 & 15.05 & 23.08 \\ \hline
            \end{tabular}} 
    \caption{Record data}
    \label{t2} 
\end{minipage}
\fi